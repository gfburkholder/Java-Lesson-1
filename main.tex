\documentclass[14pt]{extreport}%
\usepackage[Conny]{fncychap}
\usepackage{amsfonts}
\usepackage{fancyhdr}
\usepackage[usenames, dvipsnames]{color}
\usepackage{comment}
\usepackage[a4paper, top=2.5cm, bottom=2.5cm, left=2.2cm, right=2.2cm]%
{geometry}


\usepackage{titlesec}
\titleformat{\section}
{\color{Maroon}\normalfont\Large\bfseries}
{\color{Maroon}\thesection}{1em}{}[{\titlerule[0.8pt]}]

\renewcommand{\chaptername}{Unit}

\DeclareFixedFont{\ttb}{T1}{txtt}{bx}{n}{12} % for bold
\DeclareFixedFont{\ttm}{T1}{txtt}{m}{n}{12}  % for normal

\usepackage{listings}
\usepackage{color}

\definecolor{dkgreen}{rgb}{0,0.6,0}
\definecolor{gray}{rgb}{0.5,0.5,0.5}
\definecolor{mauve}{rgb}{0.58,0,0.82}

\lstset{frame=tb,
  language=Java,
  aboveskip=3mm,
  belowskip=3mm,
  showstringspaces=false,
  columns=flexible,
  basicstyle={\small\ttfamily},
  numbers=none,
  numberstyle=\tiny\color{gray},
  keywordstyle=\color{blue},
  commentstyle=\color{dkgreen},
  stringstyle=\color{mauve},
  breaklines=true,
  breakatwhitespace=true,
  tabsize=3
}
% Python environment
\lstnewenvironment{python}[1][]{\pythonstyle 
\lstset{#1}}{}
% Python for external files
\newcommand\pythonexternal[2][]{{\pythonstyle
\lstinputlisting[#1]{#2}}}
% Python for inline
\newcommand\pythoninline[1]{{\pythonstyle\lstinline!#1!}}


\titlespacing*{\chapter}{0pt}{-50pt}{40pt}

\begin{document}

% TITLE
% number following chapter should be (UNIT # - 1)
\setcounter{chapter}{0}
\chapter{\Large{\textbf{Lesson 1}: Intro to Java}}

% formatting parameter, don't change
\vspace{-10pt}

\section*{LEARNING OBJECTIVES}
\begin{itemize}
    \item Variables and Data Types \item Control Flow \item Scope \item Arrays
    
\end{itemize}

\section*{KEY CONCEPTS}
\subsection*{Variables and Data Types}
\begin{itemize}
    \item Statically Typed Language
    \item int, float, boolean, char 
\end{itemize}

\subsection*{Control Flow }
\begin{itemize}
    \item conditionals and boolean logic 
    \item loops
\end{itemize}

\subsection*{Scope}
\begin{itemize}
    \item block scope
    

    
\end{itemize}
\subsection*{Arrays }
\begin{itemize}
    \item Arrays
    \item ArrayLists
\end{itemize}


    
\section*{Variables}     
\subsection*{Data Types in Java}
There are 8 primitive data types in Java, for now we will focus on 4:
\begin{enumerate}
    \item \textbf{int}
    \begin{enumerate}
        \item short for integer, which is any positive or negative whole number such as 8, 25, 0, -16, 100
    \end{enumerate}
    \item \textbf{float} 
    \begin{enumerate}
        \item floats are numbers with decimal points such as 4.3, -6.89, 15.0
    \end{enumerate}
    \item \textbf{boolean} 
    \begin{enumerate}
        \item booleans have only two possible values: \textbf{true} or \textbf{false}
    \end{enumerate}
    
    \item \textbf{char}
    \begin{enumerate}
        \item char is short for character are includes any character on your keyboard like "a", "x", "@". A bunch of characters together surrounded by quotation marks forms a \textbf{String} (another Data Type) such as "HelloWorld123"
    \end{enumerate}
\end{enumerate}


\subsection*{Declaring Variables in Java}
here are a couple examples of creating and assigning values to variables:
\begin{lstlisting}
//setting a variable equal to an integer
int myInt = 36;

//setting a variable equal to a float
float myFloat = 84.987;

//setting a variable equal to a boolean, unlike python true and false aren't capitalized
boolean myBool = true;

//setting a variable equal to a String
String myWord = "hello";
\end{lstlisting}

There are a few things going on here, first, notice the syntax; when declaring a variable the first thing you must tell the computer is what \textit{type} the variable is going to be, then you name the variable, and then you set it equal to the value you want it to refer to, and then finally you end the statement with a semicolon \textbf{;} (every statement must have a semicolon at the end).

\begin{lstlisting}
int x = "hello";
String y = 45;
\end{lstlisting}

Types are very important and strict in Java, the above code will not run because the declared type of the variable (on the left side) does not match the type of the value assigned on the right side

\begin{lstlisting}
int x = 58;
x = 4;
x = 78;
x = 34;
System.out.println(x);
\end{lstlisting}
Once you declare the type and value of a variable, you can change the value as many times as you want and you can omit the type declaration for each new assignement after the first one because the computer already knows what the type of the variable is. Also notice the syntax for printing the variable x at the end. What value do you think this will print?

\begin{lstlisting}
int x = 58;
x = "hello";
\end{lstlisting}

The code above will result in an error. Java is what is called a \textbf{Statically Typed Language} which means that the type of a variable cannot be changed after it is first declared. The code gives an error because we initially declared x to equal an int and then later tried to change the value of x to a String.
\section*{Boolean Logic }
Now that we know how to create and store data in variables, we want to know how we can actually use them in meaningful ways. This is where \textit{contol flow} comes in.

We have already seen that booleans can have only two values: \textit{true} or \textit{false}. When we combine these values with boolean operators like \textbf{and} / \textbf{or}, we get a single overall boolean value. \\* \\*
\textbf{\textit{Boolean operators syntax}}: 
\begin{lstlisting}
true && true // and operator
true || false // or operator
!true // not operator
\end{lstlisting}

\subsection*{Problem}
\begin{lstlisting}
boolean A = true;
boolean B = true;
boolean C = false;
boolean D = true;
boolean E = false;
int outside_temp = 32;
int freezing_point = 32;
int body_temp = 97;
int coffee_temp = 120;

System.out.println(A && B);  //Q1
System.out.println((A || E) && D);  //Q2
System.out.println(A || B || D || C);  //Q3
System.out.println(A && B && D && C);  //Q4
System.out.println(outside_temp <= freezing_point);  //Q5
System.out.println(( coffee_temp > body_temp ) && E);  //Q6
\end{lstlisting}

\subsection*{Solution}
\textbf{Q1. True}. (A and B) is the same as (True and True). \\*
\textbf{Q2. True}. (A or E) evaluates to True, so we are left with (True or True). \\*
\textbf{Q3. True}. We only need one of these to be True for the whole logical statement to be True. \\*
\textbf{Q4. False}. We need all of these to be True for the whole logical statement to be True. Since C is false, the result is False. \\*
\textbf{Q5. True}. Remember that comparison operators also give a Boolean value
once they have been evaluated. 32 is less than or equal to 32. \\*
\textbf{Q6. False}. While coffee temp is greater than body temp, E is false. (True
and False) evaluates to False. \\* \\* \\*

\section*{Conditionals }
Now that we now how to evaluate boolean statements, we can use these statements to control on the computer moves through a program and to tell it to only execute certain blocks of code if some condition is met.

\subsection*{if statements}
Lets say we have a program that allows us to change how many cookies we have. We then want to increase that number by 5 and then tell us how
many we will have at the end.

\begin{lstlisting}
int cookies = 50;
cookies += 5;
System.out.println("Total cookies: " + Integer.toString(cookies));
\end{lstlisting} 
But what if they already have enough cookies? We don't have to give them
any more so that we have more to share with others. 10 sounds like it'll be
enough. So, if they have more than 10 cookies, we won't give them anymore.
Its pretty easy to translate this idea to code:
\begin{lstlisting}
int cookies = 50;

if (cookies <= 10){
  cookies += 5;
}

System.out.println("Total cookies: " + Integer.toString(cookies));
\end{lstlisting}
Here, we used an if statement to determine whether or not to give more cookies. Notice the syntax:
\begin{lstlisting}
if ( condition ){
   code to be executed when condition is met;
}
\end{lstlisting}
The keyword \textbf{if} is followed by parentheses containing a condition (that is either true or false), if that condition is true, then the code inside of the if statement is executed. Notice how we don't need a semicolon at the end of the \textit{if (condition) } statement, instead, we use a curly bracket to denote the start of the if block and another curly bracket to denote the end of the if statement. Unlike in python, the whitespace indentation is unnecessary, the curly brackets tell the computer what is inside the if block, but for the sake of readability, it is good practice to still indent whatever is inside the if block.

\subsection*{else if statements}
Our cookie program only takes one condition into account: if they have more
than 10 cookies. Another case is if they have too many, say more than 1000?
If this is true, well take away 100 so there's more for everyone!

\begin{lstlisting}
int cookies = 5000;

if (cookies <= 10){
  cookies += 5;
}

else if (cookies >= 1000){
  cookies -= 100;
}

System.out.println("Total cookies: " + Integer.toString(cookies));
\end{lstlisting}{}
Reading this code
block in English, it sounds like: if the number of cookies is less than or equal
to 10, increment by 10. Otherwise, if the number of cookies is greater than or equal
to 1000, subtract 100. Always print the total number of cookies.

\subsection*{else statements}
Finally, you may have noticed that our program does not account for all cases. We only look at times when they have under 10 or over 1000 cookies. Fortunately, Python (and many other programming languages) have a catch-all sort of statement, called else. This line will execute if and only if none of the above conditionals are evaluated to True. Lets say that in our program, we want to give 10 cookies if we execute the else block.

\begin{lstlisting}
int cookies = 100;

if (cookies <= 10){
  cookies += 5;
}

else if (cookies >= 1000){
  cookies -= 100;
}

else{
  cookies += 10;
}

System.out.println("Total cookies: " + Integer.toString(cookies));
\end{lstlisting}{}

So, if the top two conditions fail, we reach the else statement. Reading this code block in English, it sounds like: if the number of cookies is less than or equal to 10, increment by 10. Else, if the number of cookies is greater than or equal to 1000, subtract 100. Else, add 10. Always print the total number of cookies.

\subsection*{structuring if and else if blocks}
It is important to be able to distinguish between the following code blocks:
\begin{lstlisting}
int cookies = 1100;

if (cookies >= 10){
  cookies += 5;
}

else if (cookies >= 1000){
  cookies -= 100;
}

System.out.println("Total cookies: " + Integer.toString(cookies));
\end{lstlisting}{}

\begin{lstlisting}
int cookies = 1100;

if (cookies >= 10){
  cookies += 5;
}

if (cookies >= 1000){
  cookies -= 100;
}

System.out.println("Total cookies: " + Integer.toString(cookies));
\end{lstlisting}{}

Lets assume that cookies = 1100. In the top block of code, both the if and else if conditions are met, but only the first executes because the else if is reached only if the first if statement fails. At the end of the top block of code, cookies = 1105. However, in the bottom code block, both
conditions are again met, and both mathematical operations (plus 5, minus
100) are executed because the if statements are read in order. At the end of
the top block of code, cookies = 1005. The computer doesn't know
that the two if statements are related.


\section*{ACTIVITY \#1} 
\subsection*{Problem}
Description of problem.
\subsection*{Solution}
Description of solution.\\
\begin{python}
#put python code here

\end{python}

\section*{ACTIVITY \#2} 
\subsection*{Problem}
Description of problem.

\subsection*{Solution}
Description of solution.\\
\begin{python}
#put python code here
\end{python}

\section*{ACTIVITY \#3} 
\subsection*{Problem}
Description of problem.
\subsection*{Solution}
Description of solution. \\
\begin{python}
#put python code here
\end{python}

\section*{ADDITIONAL PRACTICE} 
\subsection*{Problem}
Description of problem.
\subsection*{Solution}
Description of solution. \\
\begin{python}
#put python code here
\end{python}

\section*{MORE ADVANCED PRACTICE} 
\subsection*{Problem}
Description of problem.
\subsection*{Solution}
Description of solution. \\
\begin{python}
#put python code here
\end{python}

\section*{RESOURCES}
\begin{itemize}
    \item Resource 1
    \item Resource 2
\end{itemize}

\section*{CHECK-IN}
\begin{enumerate}
    \item Question 1
    \item Question 2
\end{enumerate}

\section*{HOMEWORK}

\end{document}
